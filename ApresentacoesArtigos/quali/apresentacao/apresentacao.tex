\documentclass[10pt]{beamer}
\setbeamerfont{structure}{family=\rmfamily} 
\usepackage{amsthm}
\usepackage[brazil]{babel}
\usepackage[utf8]{inputenc}
\usepackage{graphicx}
\usepackage{graphics}
\usepackage[percent]{overpic}
\usepackage{hyperref}
\usepackage{multimedia}
\usepackage{multirow}
\beamertemplatenavigationsymbolsempty
\setbeamertemplate{blocks}[rounded][shadow=true]
\setbeamertemplate{bibliography item}[text]
\setbeamertemplate{caption}[numbered]
\usetheme{default} 
\usecolortheme{seahorse}
\mode<presentation>
{
   \setbeamercovered{transparent}
   \setbeamertemplate{items}[ball]
   \setbeamertemplate{theorems}[numbered]
   \setbeamertemplate{footline}[frame number]

}

\begin{document}
\title {\bfseries{\sc\large Estabilidade topológica e diferenciação textual
em redes de interação humana: \\
redes complexas para o participante\\
e a física antropológica
}}
\institute{
% commented by KK (put image if you want)
%\includegraphics[scale=.08]{Amrita.jpg}\medskip\\ 
	Exame de Qualificação\\
\medskip\sc{Instituto de Física de São Carlos}\\
\medskip\sc{Universidade de São Paulo}}


\author[Renato Fabbri]{\small {Orientador: Prof. Dr. Osvaldo N. O. jr.\\
		candidato: Renato Fabbri\\
}}

\date{\small 24 de Julho, 2015} 
%--------------------------------------------------------------------------%
\begin{frame}
\titlepage
\end{frame}
%---------------------------------------------------------------------------%
\section*{Roteiro}
\begin{frame}
\frametitle{Roteiro}  
\tableofcontents
\end{frame}
%---------------------------------------------------------------------------

\section{Prelúdio/Entrée/Appetizers}
\begin{frame}
\frametitle{Prelúdio/Entrée/Appetizers}

%	O que é? Traçado transdisciplinar na geografia de significação do trabalho.

Digressão teórica:
\begin{itemize}
	\item contextualiza o seminário
	\item sintoniza rigor e abstração
	\item exemplifica consequências e hipóteses
	\item incentiva a instrução, modelagem e especulação
% (ao invés da refutação)
\end{itemize}

\vspace{.7cm}

Roteiro:
\begin{itemize}
	\item complexidade (pareidolia, perversão e magia do caos)
	\item três aspectos equânimes da distribuição livre de escala
%	\item origem da desigualdade e a ubiquidade da diferença
	\item $f=\frac{R}{\lambda}$ na origem da desigualdade?
	\item metasensores
	\item ideias ideais
\end{itemize}
\end{frame}

\subsection{- complexidade, pareidolia, perversão e magia do caos}
\begin{frame}
\frametitle{- complexidade, pareidolia, perversão e magia do caos}
%http://www.m3design.com/media/2013/03/rube-goldberg.gif
%https://upload.wikimedia.org/wikipedia/commons/d/d4/Wassily_Kandinsky%2C_1939_-_Composition_X.png
\begin{figure}[!h]
    \includegraphics[width=\textwidth]{../figs/Wassily_Kandinsky,_1939_-_Composition_X}\\
    (Composition X, Kandinsky - 1939)
        \label{fig:mesam}
\end{figure}
\end{frame}

\subsection{- três aspectos equânimes da distribuição livre de escala}
\begin{frame}
\frametitle{- três aspectos equânimes da distribuição livre de escala}
\begin{figure}
	\begin{columns}
		\column{.1\textwidth}
		\includegraphics[width=\textwidth]{../figs/Triple-Spiral-Symbol-heavystroked}
		\column{.9\linewidth}
ao menos para as redes humanas, os participantes transitam continuamente entre os setores de hubs, intermediários e periféricos. Exceção para redes muito pequenas
	\end{columns}
\end{figure}



\begin{figure}
	\begin{columns}
		\column{.2\textwidth}
		\includegraphics[width=\textwidth]{../figs/F1large}
		\column{.9\linewidth}
o tempo disponível para cada participante é o mesmo, e a distribuição nas diferentes redes praticamente sempre implica no mesmo participante ser hub, intermediário e periférico em diferentes redes. Por exemplo: somos hubs em algumas redes de nossos trabalhos e de nossas famílias, periféricas em algumas redes de trabalhos e famílias de conhecidos
	\end{columns}
\end{figure}



\begin{figure}
	\begin{columns}
		\column{.3\textwidth}
		\includegraphics[width=\textwidth]{../figs/rt}
		\column{.7\linewidth}
		o meio aloca a mesma quantidade de recurso ao longo da conectividade em uma rede livre de escala. Ou seja, há uma distribuição equânime de recursos ao longo da conectividade: $f=\frac{v}{\lambda} \Rightarrow log(f)=-1 log(\lambda) + log(v) \Rightarrow \alpha=1$. Ao mesmo tempo, $f . \lambda = v = constante$.
		\label{fig:example right}
	\end{columns}
\end{figure}

\end{frame}

%%%%%%%%%%%% TTM





\subsection{- $f=\frac{v}{\lambda}$ está na origem da desigualdade?}
\begin{frame}
\frametitle{$f=\frac{v}{\lambda}$ está na origem da desigualdade?}
\begin{equation}
	f=\frac{v}{\lambda_1 \lambda_2},\;\;\; \lambda \approx \lambda_1 \approx \lambda_2 \Rightarrow log(f) = -2log(\lambda) + log(v)
\end{equation}

Em uma rede, temos essencialmente $E$ arestas e $N$ vértices.
Assumindo linearidade:
\begin{equation}
	Recursos=\alpha N + \beta E \approx \beta E = \beta N \frac{E}{N} = \beta \frac{N \overline{k}}{2}
\end{equation}

onde $\overline{k}$ é o grau médio. O termo $\alpha N$ pode ser descartado se assumirmos interesse em rede de interação, onde $N$ participantes que não interagem pode ser considerado o estado de mínimo/nulo emprego de recursos pelo sistema (para a interação de interesse e observável).

\begin{equation}
	f=\frac{v}{Recurso_i}=\frac{v}{Recurso_i}=\frac{v}{\beta E_i}=\frac{2}{\beta}\frac{v}{N_i . \overline{k_i}} \equiv \frac{v}{\lambda_1 \lambda_2}
\end{equation}
% ,\;\;\; \lambda \approx \lambda_1 \approx \lambda_2 \Rightarrow log(f) = -2log(\lambda) + log(v)

$N$ e $\overline{k}$ são sempre diretamente proporcionais à quantidade de recursos alocados se fixados um deles ($N$ ou $\overline{k}$).

\end{frame}

\subsection{- metasensores \;\; - ideias ideais}
\begin{frame}
\frametitle{- metasensores}

Cada vértice subsiste no tempo, é um observador, um sensor, uma unidade de processamento de informação e matéria. As arestas correspondem ao relacionamento entre os dois sensores, criando um sensor composto, dual. A associação sucessiva de sensores implica em um meta-sensor. Considerações:

\begin{itemize}
	\item teoria básica de complexidade sugere isso: sistemas adaptativos para eficiência e sobrevivência, processam informação, interagem entre si, por vezes se replicam (e.g. idosos em uma praça).
	\item densidade média de esferas no caso tridimensional: $f=\frac{v}{\lambda_1 \lambda_2 \lambda_3}\equiv \frac{3}{4 \pi} \frac{v}{r^3}$
	\item extensão da percepção também segue a lei de potência: $\Delta_8=2^{}$
	\item as arestas são criadas através da alocação de tempo pelos participantes. O recurso básico é tempo. A rede é um sensor de sinal temporal quadrático?
	\item o universo é o sensor máximo?
\end{itemize}

Exemplo do sensor sobre a Unicamp (geração de conhecimento, ganha amigos, concentração segundo lei de potência, motor dialético, sensor colorido).
\end{frame}

%\subsection{- ideias ideais}
\begin{frame}
\frametitle{- ideias ideais}
Teoria física das ideias.

Uma ideia ideal é um objeto físico idealizado, como uma superfície ideal.

\begin{figure}[!h]
    \includegraphics[width=.3\textwidth]{../figs/ideia2}
    \hspace{2cm}
    \includegraphics[width=.3\textwidth]{../figs/variasUma}
\end{figure}
\begin{figure}[!h]
    \includegraphics[width=.3\textwidth]{../figs/planoIdeias}
    \hspace{2cm}
    \includegraphics[width=.3\textwidth]{../figs/boot1}
\end{figure}



\end{frame}

\section{Introdução}
\subsection{- redes complexas e de interação humana}
\begin{frame}
\frametitle{Introdução}
São $10^{80}$ {\bf átomos no universo observável},
uma referência de escala.
Considere o número $N$ de pessoas necessário para
haver {\bf mais redes possíveis do que átomos no universo}.
Cada aresta é uma variável de Bernoulli fruto de cada
par de vértices: a aresta pode estar presente ou não.

\begin{align}
2^{N \choose 2} > 10^{80} \Rightarrow 
log_2[2^{N \choose 2}] > log_2(10^{80}) \Rightarrow
{N \choose 2} > \frac{log_{10}(10^{80})}{log_{10}2} \Rightarrow \nonumber\\
\Rightarrow \frac{N.(N-2)}{2} > \frac{80}{log_{10}2} \Rightarrow
	N > 23,5988 \;\;\;\;\;\;\;\;\;\;\;\;\;\;\;\;\;\;\;\;\;
	\nonumber
\end{align}


Isso justifica a utilidade de paradigmas para as redes,
e das medidas para cada vértice e para a rede,
instrumental para as {\bf redes complexas}, incluindo
as {\bf redes de interação humana}.
%[fundo é Vídeo do versinus e imagens, redes minhas mesmo]

{\bf Sistemas complexos} $\Rightarrow$ sistemas que processam informação,
exibem mecanismos adaptativos e de subsistência.
Constituído de várias partes cuja interação implica
em comportamento emergente. Sistema intrínseco ao meio
em que subsiste e integrado a outros sistemas dinâmicos.
\end{frame}


\begin{frame}
\frametitle{Introdução}
\begin{center}
\movie[width=0.7\textwidth,showcontrols=true]
{% placeholder = text or image
	%\begin{overpic}[width=\textwidth,grid,tics=10]{../figs/ideia2}
	\begin{overpic}[width=0.7\textwidth]{../figs/CienciasComFronteiras}
		%		 \put (00,50) {  
		%Obervação em evolução.
		%      }
	 \end{overpic}
}%
{final.avi} % video filename
%{sintel_trailer-480p.mp4} % video filename
\end{center}
\end{frame}



\subsection{- ontologia do trabalho \;\; - vocabulário do trabalho}
\begin{frame}
\frametitle{- ontologia do trabalho}
Formalização dada ontologia OWL com a formalização das
áreas envolvidas. 


\vspace{1cm}
\begin{figure}[!h]
    \includegraphics[width=.7\textwidth]{../figs/legenda}
\end{figure}

\vspace{1cm}

Situar alguns indivíduos na ontologia, que são alguns dos feitos.


Contemplar de redes de interação humana em evolução temporal até Redes complexas, estatística e ignorância?

\end{frame}
%\subsection{- vocabulário do trabalho}
\begin{frame}
\frametitle{- vocabulário do trabalho}
Vocabulário SKOS, compreendendo principais definições, assim como 
os casos de ambiguidades, sinônimos e formas preferidas.
Complementa as definições
e conceitualização da OT. Termos especiais:
\begin{itemize}
	\item Física Antropológica
	\item Estabilidade temporal
	\item Setorialização de Erdös, setores de Erdös
	\item Rede complexa
\end{itemize}
\end{frame}

\begin{frame}
\frametitle{- objetivos e justificativas}
O objetivo geral é comprovar e aprofundar a utilização das redes complexas
pelo participante. Objetivos específicos:
\begin{itemize}
	\item melhor compreensão sobre nossas estruturas sociais
	\item entrega de um legado tecnológico para registrar e disponibilizr os desenvolvimentos
	\item entrega de um legado em dados sociais, de fontes heterodoxas e apropriadas para a análise conjunta
	\item delineio da prática de estudo das próprias estruturas sociais 
\end{itemize}

\vspace{.5cm}

A justificativa principal é que há um hiato proeminente
entre esta frente científica e tecnológica e a utilização que
os participantes destas redes fazem dela. Justificaticas secundárias são:
\begin{itemize}
	\item A área é recente e reconhecidamente útil e potente.
	\item As instituições já fazem uso destes conhecimentos
		há séculos (talvez milênios). A frente civil está começando a surgir.
	\item Tenho um perfil adequado à transdisciplinaridade envolvida.
\end{itemize}

\end{frame}



\section{Materiais: dados de email, Facebook, Twitter, Participa.br, AA, IRC}
\begin{frame}
\frametitle{Materiais}
\begin{itemize}
	\item Mensagens de e-mail, com horário de envio, ID da mensagem, ID da mensagem anterior na thread se existente, texto do título e corpo. 
	\item Redes de Facebook: redes GML ou GDF geralmente baixadas do Graphviz, mas também raspadas de minha própria conta. as únicas informações da rede são: nome e ID de cada amigo, aresta entre cada par de amigos que forem amigos entre si. Nas redes de interação constam arestas dirigidas. As redes eram de pessoas que me mandavam elas de suas contas ou minhas pessoais ou de grupos dos quais participava.
	\item Participa.br: redes de amizade e de interação, texto de postagens, comentários, etc.
	\item Twitter: milhões de tweets permitiram observação contínua de redes de interação (retweet), relacionamentos por vocabulário e hashtags, e padrões do vocabulário em si.
	\item Materiais coletados com entrevistas e oficinas com especialistas.
	\item Estruturas semânticas e dados etiquetados.
\end{itemize}
\end{frame}



\section{Métodos}
\begin{frame}
\frametitle{Métodos}
\begin{itemize}
	\item Estatística circular
	\item (obtenção das redes de interação)
	\item Setorialização de Erdös
	\item PCA de medidas topológicas
	\item Testes de Kolmogorov-Smirnoff dos textos
	\item Web semântica
	\item Audiovisualização de dados
	\item Considerações tipológicas e humanísticas
\end{itemize}
\end{frame}

\begin{frame}
\subsection{- estatística circular \;\; - redes de interação \;\; - setorialização de Erdös}
\frametitle{- estatística circular}
Com $m_n=\frac{1}{N}\sum_{i=1}^N z_i^n$ o n-ésimo momento:
\begin{align}\label{eq:cmom}
    R_n&=|m_n| \nonumber \\
    \theta_\mu&=Arg(m_1) \\
    \theta_\mu'&=\frac{period}{2\pi} \theta_\mu \nonumber
\end{align}

\begin{align}
    Var(z)&=1 - R_1 \nonumber\\
    S(z)&= \sqrt{-2\ln(R_1)}\\
    \delta(z)&=\frac{1-R_2}{2 R_1^2} \nonumber
\end{align}

Usamos também $\frac{b_h}{b_l}$ entre a maior $b_h $ e a menor $b_l$ incidência nos histogramas.

\end{frame}
\begin{frame}
%\subsection{- redes de interação}
\frametitle{- redes de interação}
\begin{figure}[!h]
    \centering
    \includegraphics[width=0.5\textwidth]{../figs/criaRede__}
\end{figure}
\end{frame}
\begin{frame}
%\subsection{- setorialização de Erdös \;\;\;\;\;  - PCA de medidas topológicas}
%\subsection{- setorialização de Erdös \;\;\;\;\;  - PCA de medidas topológicas}
\frametitle{- setorialização de Erdös}

\begin{figure}[!h]
    \centering
    \includegraphics[width=.7\textwidth]{../figs/fser_}
        \label{fig:setores}
\end{figure}

\begin{equation}\label{criterio2}
    \sum_{x=k_i}^{k_j} \widetilde{P}(x) < \sum_{x=k_i}^{k_j} P(x) \Rightarrow \text{i é intermediário}
\end{equation}

\begin{equation}
    P(k)=\binom{2(N-1)}{k}p_e^k(1-p_e)^{2(N-1)-k}
\end{equation}
onde 
\centering
$p_e=\frac{z}{N(N-1)}$


\end{frame}
\begin{frame}
%\subsection{- PCA de medidas topológicas}
\subsection{- PCA de medidas topológicas \;\; - Kolmogorov-Smirnoff para textos}
\frametitle{- PCA de medidas topológicas}
Médias e desvios das medidas $j$ nas componentes $k$ fruto de $L$ observações $l$:
\begin{align}\label{eq:pca}
\mu_{V'}[j,k]   &=\frac{\sum_l^L V'[j,k,l]}{L}\nonumber\\
\sigma_{V'}[j,k]&=\sqrt{\frac{(\mu_{V'}-V'[j,k,l])^2}{L}}\\\nonumber
\mu_{D'}[k]&=\frac{\sum_l^L D'[k,l]}{L}\\\nonumber
\sigma_{D'}[k]&=\sqrt{\frac{(\mu_{D'}-D'[k,l])^2}{L}}
\end{align}

Foco nas medidas de centralidade e clusterização mais usuais. 
Inseridas medidas de simetria potencialmente novas.

\end{frame}
\begin{frame}
%\subsection{- Teste de Kolmogorov-Smirnoff para textos}
\frametitle{- Teste de Kolmogorov-Smirnoff para incidencias em textos}
\begin{equation}\label{eq:ks}
D_{n,n'} > c(\alpha)\sqrt{\frac{n+n'}{nn'}} \Rightarrow F_{1,n} \neq F_{2,n'}
\end{equation}
\vspace{2cm}
\begin{equation}\label{eq:ks}
c(\alpha) < \frac{D_{n,n'}}{\sqrt{\frac{n+n'}{nn'}}} = c'(\alpha)
\end{equation}
\end{frame}
\begin{frame}
\subsection{- web semântica \;\; - audiovisualização de dados}
\frametitle{- web semântica / dados ligados}
	Recomendação da W3C para a formalização de conceitualizações, o relacionamento dos dados a estas conceitualizações, o armazenamento
	de dados semanticamente enriquecidos, o relacionamento de dados de fontes diferentes, a navegação semântica e a inferência por máquina.

	Observações especiais:
\begin{itemize}
	\item Redes estáveis em nosso tecido social em certas escalas temporais e de pessoas.
		Passível de transições de fase, modificações abruptas em outras.
	\item Permite: análise conjunta de dados de diferentes fontes; desenvolvimento conceitual compartilhado.
	\item Recomendação da W3C; padrão acadêmico para dados semânticos etiquetados; melhor formato para entregar os dados para a sociedade como um legado para análise e experimentos.
	\item Pesado e um pouco complicado. Uso de ferramentas como Fuseki/Jena para facilitar os usos.
	\item Procurar uma notação mais poderosa para os dados ligados?
\end{itemize}

\end{frame}
\begin{frame}
%\subsection{- audiovisualização de dados}
\frametitle{- audiovisualização de dados}
Permite maior contato com as estruturas de interesse,
o que facilita a condução da pesquisa para questões mais
fundamentais e observáveis.

Algumas estratégias utilizadas:
\begin{itemize}
	\item versinus, imagens, animação abstrata com música, sonificações.
	\item roteiros automatizados de realização de arte social
	\item freakcoding (subgênero do livecoding)
	\item arte governamental
\end{itemize}
\end{frame}

\begin{frame}
\subsection{- considerações tipológicas e humanísticas}
\frametitle{- considerações tipológicas e humanísticas}
\begin{itemize}
	\item redes de seres humanos
	\item consideração do fator estigmatizante
	\item apreciação do meio em que a rede é observada
	\item experimentos percolatórios
	\item física antropológica
\end{itemize}
\end{frame}

\section{Resultados}
\begin{frame}
\frametitle{Resultados}
\begin{itemize}
	\item Estabilidade temporal
	\item Diferenciação textual
	\item Iniciação da nuvem brasileira de dados ligados participativos
	\item Aparato em software
	\item Beneficiamento
	\item Ideias ideais (teoria física das ideias)
\end{itemize}
\end{frame}

\subsection{- estabilidade temporal e topológica \;\; - diferenciação textual}
\begin{frame}
\frametitle{- estabilidade temporal e topológica}
\begin{itemize}
	\item Medidas circulares praticamente iguais para todas as listas e em todas as escalas de segundos a semestres.
	\item Constância dos tamanhos dos setores de Erdös, compatível com as expectativas da literatura. Ainda não achei formalização para esta expectativa e talvez esta seja a primeira.
	\item Estabilidade das componentes principais. Prevalência da centralidade, seguida da simetria e então clusterização dos participantes.
	\item Tipologia não estigmatizante de participante. Tipologia de rede.
\end{itemize}
\end{frame}
\begin{frame}
\frametitle{- estabilidade temporal e topológica}

\vspace{-.2cm}
\begin{figure} 
	\centering
	\includegraphics[width=.75\textwidth]{/home/r/repos/stabilityInteraction/figs/InText-WLAU-S1000}
\end{figure}
\vspace{-.2cm}

	\begin{minipage}[t]{0.48\linewidth}
\begin{table}
	\tiny
\begin{center} 
	\begin{tabular}{| p{.2cm} || p{.37cm} | p{.37cm} | p{.37cm} | p{.37cm} | p{.37cm} | p{.37cm} |}\hline 
  & 1h & 2h & 3h & 4h & 6h & 12h \\\hline 
 0h  &\multirow{1}{*}{3.66 }   &\multirow{2}{*}{6.42 }   &\multirow{3}{*}{8.20 }   &\multirow{4}{*}{9.30 }   &\multirow{6}{*}{10.67 }   &\multirow{12}{*}{33.76 }  \\\cline{2-2} 
 1h  &\multirow{1}{*}{2.76 }   &   &   &   &   &  \\\cline{2-2}\cline{3-3} 
 2h  &\multirow{1}{*}{1.79 }   &\multirow{2}{*}{2.88 }   &   &   &   &  \\\cline{2-2}\cline{4-4} 
 3h  &\multirow{1}{*}{1.10 }   &   &\multirow{3}{*}{2.47 }   &   &   &  \\\cline{2-2}\cline{3-3}\cline{5-5} 
 4h  &\multirow{1}{*}{0.68 }   &\multirow{2}{*}{1.37 }   &   &\multirow{4}{*}{3.44 }   &   &  \\\cline{2-2} 
 5h  &\multirow{1}{*}{0.69 }   &   &   &   &   &  \\\cline{2-2}\cline{3-3}\cline{4-4}\cline{6-6} 
 6h  &\multirow{1}{*}{0.83 }   &\multirow{2}{*}{2.07 }   &\multirow{3}{*}{4.35 }   &   &\multirow{6}{*}{23.09 }   &  \\\cline{2-2} 
 7h  &\multirow{1}{*}{1.24 }   &   &   &   &   &  \\\cline{2-2}\cline{3-3}\cline{5-5} 
 8h  &\multirow{1}{*}{2.28 }   &\multirow{2}{*}{6.80 }   &   &\multirow{4}{*}{21.03 }   &   &  \\\cline{2-2}\cline{4-4} 
 9h  &\multirow{1}{*}{4.52 }   &   &\multirow{3}{*}{18.75 }   &   &   &  \\\cline{2-2}\cline{3-3} 
 10h  &\multirow{1}{*}{6.62 }   &\multirow{2}{*}{\textbf{14.23} }   &   &   &   &  \\\cline{2-2} 
 11h  &\multirow{1}{*}{\textbf{7.61} }   &   &   &   &   &  \\\cline{2-2}\cline{3-3}\cline{4-4}\cline{5-5}\cline{6-6}\cline{7-7} 
 12h  &\multirow{1}{*}{6.44 }   &\multirow{2}{*}{12.48 }   &\multirow{3}{*}{\textbf{18.95}}   &\multirow{4}{*}{\textbf{25.05\;}}&\multirow{6}{*}{\textbf{37.63}}   &\multirow{12}{*}{\textbf{66.24}}  \\\cline{2-2} 
 13h  &\multirow{1}{*}{6.04 }   &   &   &   &   &  \\\cline{2-2}\cline{3-3} 
 14h  &\multirow{1}{*}{6.47 }   &\multirow{2}{*}{12.57 }   &   &   &   &  \\\cline{2-2}\cline{4-4} 
 15h  &\multirow{1}{*}{6.10 }   &   &\multirow{3}{*}{18.68 }   &   &   &  \\\cline{2-2}\cline{3-3}\cline{5-5} 
 16h  &\multirow{1}{*}{6.22 }   &\multirow{2}{*}{12.58 }   &   &\multirow{4}{*}{23.60 }   &   &  \\\cline{2-2} 
 17h  &\multirow{1}{*}{6.36 }   &   &   &   &   &  \\\cline{2-2}\cline{3-3}\cline{4-4}\cline{6-6} 
 18h  &\multirow{1}{*}{6.01 }   &\multirow{2}{*}{11.02 }   &\multirow{3}{*}{15.88 }   &   &\multirow{6}{*}{28.61 }   &  \\\cline{2-2} 
 19h  &\multirow{1}{*}{5.02 }   &   &   &   &   &  \\\cline{2-2}\cline{3-3}\cline{5-5} 
 20h  &\multirow{1}{*}{4.85 }   &\multirow{2}{*}{9.23 }   &   &\multirow{4}{*}{17.59 }   &   &  \\\cline{2-2}\cline{4-4} 
 21h  &\multirow{1}{*}{4.38 }   &   &\multirow{3}{*}{12.73 }   &   &   &  \\\cline{2-2}\cline{3-3} 
 22h  &\multirow{1}{*}{4.06 }   &\multirow{2}{*}{8.36 }   &   &   &   &  \\\cline{2-2} 
 23h & \multirow{1}{*}{4.30 }  & & & & & \\\cline{2-2}\cline{3-3}\cline{4-4}\cline{5-5}\cline{6-6}\cline{7-7} 
 \hline\end{tabular} 
 \end{center}
\end{table}


\end{minipage}
\hfill%
	\begin{minipage}[t]{0.48\linewidth}
\begin{table}[!h]
		\tiny
\begin{center}
\begin{tabular}{| l | c | c | c | c | c | c |}\cline{2-7}
\multicolumn{1}{c|}{} & \multicolumn{2}{c|}{PC1}          & \multicolumn{2}{c|}{PC2} & \multicolumn{2}{c|}{PC3}  \\\cline{2-7}\multicolumn{1}{c|}{} & $\mu$            & $\sigma$ & $\mu$         & $\sigma$ & $\mu$ & $\sigma$  \\\hline
$cc$ & 0.89  & 0.59  & 1.93  & 1.33  & 21.22  & 2.97 \\\hline
$s$ & 11.71  & 0.57  & 2.97  & 0.82  & 2.45  & 0.72 \\
$s^{in}$ & 11.68  & 0.58  & 2.37  & 0.91  & 3.08  & 0.78 \\
$s^{out}$ & 11.49  & 0.61  & 3.63  & 0.79  & 1.61  & 0.88 \\
$k$ & 11.93  & 0.54  & 2.58  & 0.70  & 0.52  & 0.44 \\
$k^{in}$ & 11.93  & 0.52  & 1.19  & 0.88  & 1.41  & 0.71 \\
$k^{out}$ & 11.57  & 0.61  & 4.34  & 0.70  & 0.98  & 0.66 \\
$bt$ & 11.37  & 0.55  & 2.44  & 0.84  & 1.37  & 0.77 \\\hline
$asy$ & 3.14  & 0.98  & 18.52  & 1.97  & 2.46  & 1.69 \\
$\mu_{asy}$ & 3.32  & 0.99  & 18.23  & 2.01  & 2.80  & 1.82 \\
$\sigma_{asy}$ & 4.91  & 0.59  & 2.44  & 1.47  & 26.84  & 3.06 \\
$dis$ & 2.94  & 0.88  & 18.50  & 1.92  & 3.06  & 1.98 \\
$\mu_{dis}$ & 2.55  & 0.89  & 18.12  & 1.85  & 1.57  & 1.32 \\
$\sigma_{dis}$ & 0.57  & 0.33  & 2.74  & 1.63  & 30.61  & 2.66 \\\hline\hline
$\lambda$ & 49.56  & 1.16  & 27.14  & 0.54  & 13.25  & 0.95 \\
\hline\end{tabular}
\end{center}
	\end{table}


	\end{minipage}





\end{frame}

\begin{frame}
%\subsection{- diferenciação textual}
\frametitle{- diferenciação textual}
\begin{itemize}
	\item O texto produzido por cada setor de Erdös é extremamente diferente um do outro, maiores do que texto produzido por redes diferentes ou mesmo por setores iguais de redes diferentes.
	\item Hubs produzem mais adjetivos. Periféricos mais substantivos, etc.
	\item Formalização de dados ligados a partir de dados relacionais participativos.
	\item Correlações não triviais.
	\item Combinação moderada de medidas topológicas e textuais; prevalência (não extrema) de componentes de texto ou topologia.
	\item Constância da existência - incidência nos textos observados.
\end{itemize}
\end{frame}


\begin{frame}
\frametitle{- diferenciação textual}

\begin{table}[H]
\centering
\tiny
\begin{tabular}{|l||c|c|c|c|c|c|}\hline
$\alpha$    & 0.1  & 0.05 & 0.025 & 0.01 & 0.005 & 0.001 \\\hline
$c(\alpha)$ & 1.22 & 1.36 & 1.48  & 1.63 & 1.73  & 1.95  \\\hline
\end{tabular}
\end{table}
\vfill
\begin{minipage}[t]{0.4\linewidth}
\begin{table}
    \tiny
\begin{center} 
\setlength{\tabcolsep}{.26667em}
  \begin{tabular}{|l|| c|c|c|}\hline
list$\setminus$measure & H-P & H-I & I-P \\\hline
CPP & 5.58 & 2.54 & 7.82 \\\hline
LAD & 7.67 & 2.07 & 8.35 \\\hline
LAU & 6.23 & 1.63 & 5.98 \\\hline
ELE & 3.42 & 0.77 & 2.81 \\\hline
  \end{tabular}
 \end{center}
\end{table}
\end{minipage}
%\hfill%
\hspace{1cm}%
	\begin{minipage}[t]{0.4\linewidth}
\begin{table}[!h]
\tiny
\begin{center}
\setlength{\tabcolsep}{.06667em}
  \begin{tabular}{|l|| c|c|c|c|c|c|}\hline
& CPP-LAD & CPP-LAU & CPP-ELE & LAD-LAU & LAD-ELE & LAU-ELE \\\hline
P & 1.35 & 4.05 & 5.80 & 3.00 & 5.41 & 4.94 \\\hline
I & 1.27 & 0.78 & 4.01 & 0.84 & 3.84 & 3.94 \\\hline
H & 0.98 & 1.94 & 3.17 & 1.32 & 3.82 & 4.47 \\\hline
  \end{tabular}
\end{center}
\end{table}
\end{minipage}
\vfill
\begin{figure}[h!]
    \centering
    \includegraphics[width=.7\textwidth]{/home/r/repos/artigoTextoNasRedes/figs/kw}
\end{figure}



\end{frame}



\begin{frame}
\subsection{- dados sociais ligados \;\; - peças artísticas e mapeamentos sensoriais}
\frametitle{- dados sociais ligados}
\begin{itemize}
	\item Síntese de ontologias (OWL) e vocabulários (SKOS) de estruturas sociais. OPS, OPa, OPP, Ontologiaa, OCD, OBS, VBS.
	\item Formalização de dados ligados a partir de dados relacionais participativos.
	\item Método de construção de ontologias orientado aos dados.
\end{itemize}

\begin{figure}[h!]
    \centering
    \includegraphics[width=\textwidth]{/home/r/repos/vocabularioParticipacao/figs/obsPNPS_mesam}
\end{figure}


\end{frame}

\begin{frame}
%\subsection{- peças artísticas e mapeamentos sensoriais}
\frametitle{- peças artísticas e mapeamentos sensoriais}
\begin{itemize}
	\item Four hubs dance. Prelúdio social
	\item Versinus (linha+senóide).
	\item Outros casos: app online (PHP+python) para imagens do GMANE, sonificações.
	\item Apresentações artísticas: Crânio de Rilke (grupo de teoria crítica e fechamento do congresso internacional), Freakcoding.
\end{itemize}

\begin{center}
	% figura do freakcoding, video do preludio social
\movie[width=0.7\textwidth,showcontrols=true]
{% placeholder = text or image
	%\begin{overpic}[width=\textwidth,grid,tics=10]{../figs/ideia2}
	\begin{overpic}[width=0.7\textwidth]{../figs/fig_novela}
		%		 \put (00,50) {  
		%Obervação em evolução.
		%      }
	 \end{overpic}
}%
{mixedVideo.webm} % video filename
%{sintel_trailer-480p.mp4} % video filename
\end{center}

\end{frame}

\begin{frame}
\subsection{- software \;\; - beneficiamento \;\; - empréstimos antropológicos}
\frametitle{- software}
Pacote oficial da linguagem Python (PyPI) para compartilhamento
preciso e eficiente dos desenvolvimentos. Mais especificamente:
	\begin{itemize}
		\item observação das estabilidades topológicas e diferenciações textuais. (Gmane)
		\item Acesso à nuvem de dados participativos brasileiros, jutno aos scritps de triplificação. para análise . (Participation)
		\item Anotação automatizada e semântica de seus próprios dados virtuais provenientes do Facebook, Twitter, Diáspora, IRC, etc. (Social)
		\item Mapeamentos precisos de estruturas sociais em sonoras através da melhor qualidade de síntese. (MASS)
		\item Integração destes dados todos para análise conjunta e navegação semântica enriquecida. Realização de músicas e animações abstratas explorando estas propriedades-chave. (Percolation)
	\end{itemize}
\end{frame}

\begin{frame}
%\subsection{- beneficiamento}
\frametitle{- beneficiamento}
\begin{itemize}
	\item recomendação de recursos para enriquecimento da navegação semântica.
	\item experimentos percolatórios
	\item geração de arte a partir das estruturas sociais
	\item formalização de estruturas de governança (gerenciais) da sociedade
	\item sistema de construção de ontologias OWL orientado aos dados
	\item compreensão sobre as estruturas sociais: estabilidades, diferenciações
	\item enriquecimento semântico de dados relacionais e sua tradução para dados ligados
	\item fundamentação da origem da lei de potência nestes contextos
	\item contribuição ao cânone de tipologias humanas e sociais, que é humanístico, através da física
\end{itemize}
\end{frame}

\begin{frame}
%\subsection{- empréstimos antropológicos}
\frametitle{- empréstimos antropológicos}
\begin{itemize}
\item histórico do termo física antropológica, de Boaz a este trabalho
\item aspectos reflexivos, biográficos. Estudo e exposição de si. Diário. Leitura da curtíssima autobiografia
\item comparação entre física e antropologia. Considerações sobre uma ciência sólida
\item \emph{Social Physics} do Pentland (co-fundador e diretor do MIT Media Lab)
\end{itemize}
\end{frame}

\begin{frame}
%\subsection{- ideias ideais}
\frametitle{- ideias ideais}
%\includegraphics[scale=.23]{FlowChart.jpg}\\
\begin{itemize}
\item o que é
\item como é formada a rede de uma ideia
\item livre de escala na forma principal, possui diversos harmônicos
\item pelo mesmo raciocínio da origem da propriedade livre de escala, podem ser considerados sensores abstratos
\item ideias ideais como unificação das redes sociais formadas por indivíduos e por conceitos. Sensores imateriais de informação?
\item ontologias são Rich Clubs?
\item exemplos de 2002. Discussão de algum caso partindo do sistema axiomático
\end{itemize}
\end{frame}

\section{Conclusões}
\begin{frame}{Conclusões}
\begin{itemize}
	\item gradus ad parnassum unificando:
		\begin{itemize}
			\item  apresentação breve e instrumental da área para o indivíduo. Conceitos fundamentais. Paradigmas de redes. Sinonímias, e ambiquidades. Caracterização de redes humanas e beneficiamento para o indivíduo através de experimentos antropológicos, análise e navegação
				Apêndice com listagens úteis, como medidas, software, trabalhos de referência, etc
			\item consideração da física antropológica (ou como esta conceituação se resolver)
			\item aparato em software, protocolos e dados
		\end{itemize}
	\item tipologias com o aprofundamento das análises de estabilidade temporal e diferenciação textual
	\item legado em software, ontologias e dados
	\item estou aquém de equipe e qualidade de vida do que acredito adequado
	\item ignorância. Froteira da ilha sempre se ampliando
	\item gostarei de terminar a graduação da física aos poucos se eu tiver a oportunidade. Creio que causará bastante estímulo e isso pode ser conveniente daqui alguns anos. Talvez proficiências para titulação
\end{itemize}
\end{frame}

\subsection{- próximos passos \;\; - ideiais ideias \;\; - agradecimentos e referências}
\begin{frame}{- próximos passos}
Meu norte seria já poder estudar diretamente a cadeia de sistemas dinâmicos, do social ao universo. Com sua inércia, deve dar para tirar umas boas propriedades relacionando as frequências de sistemas complexos em escalas diferentes. As referências de escala individual ao universo observável devem implicar em propriedades.

Ou seja:
	\begin{itemize}
		\item muita coisa boa para fazer, seria bom delimitar o que vai para pós-doc ou próximo vínculo
		\item isso parece depender de alguma previsão de vínculo apropriado e de onde
		\item a prioridade é permanecer em S. Carlos, de preferência relacionado ao IFSC. Possibilidades na UFSCar, IPRJ, UnB, UFABC
		\item acho que é na academia que conseguirei me dedicar para a pesquisa podendo imergir como gosto
	\end{itemize}
\end{frame}



\begin{frame}{Referências}
  \begin{thebibliography}{99}
  \bibitem{one}
	  Renato Fabbri, \emph{Estabilidade topológica e diferenciação textual em redes de interação humana: redes complexas para o participante e a física antropológica} (monografia de qualificação). Online em
	  \url{https://github.com/ttm/tese/raw/master/ApresentacoesArtigos/quali/qualiFinal.pdf}
  \end{thebibliography}
\end{frame}

\begin{frame}
\Large
\begin{center}
 \sc {Obrigado \ldots} 
\end{center}
\end{frame}
%---------------------------------------------------------------------------%
\end{document}

