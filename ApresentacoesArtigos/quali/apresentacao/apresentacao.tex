\documentclass[10pt]{beamer}
\setbeamerfont{structure}{family=\rmfamily} 
\usepackage{amsthm}
\usepackage[brazil]{babel}
\usepackage[utf8]{inputenc}
\usepackage{graphicx}
\usepackage{graphics}
\usepackage{hyperref}
\beamertemplatenavigationsymbolsempty
\setbeamertemplate{blocks}[rounded][shadow=true]
\setbeamertemplate{bibliography item}[text]
\setbeamertemplate{caption}[numbered]
\usetheme{default} 
\usecolortheme{seahorse}
\mode<presentation>
{
   \setbeamercovered{transparent}
   \setbeamertemplate{items}[ball]
   \setbeamertemplate{theorems}[numbered]
   \setbeamertemplate{footline}[frame number]

}

\begin{document}
\title {\bfseries{\sc\large Estabilidade topológica e diferenciação textual
em redes de interação humana: \\
redes complexas para o participante\\
e a física antropológica
}}
\author[Renato Fabbri]{\small {Renato Fabbri\\
Orientador: Prof. Dr. Osvaldo N. O. jr.}}
\institute{
% commented by KK (put image if you want)
%\includegraphics[scale=.08]{Amrita.jpg}\medskip\\ 
\sc{Instituto de Física de São Carlos}\\  \medskip\sc{Universidade de São Paulo}}
\date{\small 24 de Julho, 2015} 
%--------------------------------------------------------------------------%
\begin{frame}
\titlepage
\end{frame}
%---------------------------------------------------------------------------%
\section*{Roteiro}
\begin{frame}
\frametitle{Roteiro}  
\tableofcontents
\end{frame}
%---------------------------------------------------------------------------

\section{Introdução}
\begin{frame}
\frametitle{Redes complexas e de interação humana}
Vídeo versinus, imagens.
Cada aresta uma variável de Bernoulli.
$2^{42 \choose 2}$ ultrapassa átomos no universo,
utilidade de paradigmas para as redes,
e das medidas para cada vértice e para a rede.
\end{frame}

\begin{frame}
\frametitle{Ontologia do trabalho}
diagrama da ontologia OWL com a formalização das
áreas envolvidas. Situar alguns indivíduos na ontologia,
que são alguns dos feitos.

Contemplar de redes de interação humana em evolução temporal até Redes complexas, estatística e ignorância.
\end{frame}

\section{Materiais}
\begin{frame}
\frametitle{Materiais}
\begin{itemize}
	\item Mensagens de e-mail, com horário de envio, ID da mensagem, ID da mensagem anterior na thread se existente, texto do título e corpo. 
	\item Redes de Facebook: redes GML ou GDF geralmente baixadas do Graphviz, mas também raspadas de minha própria conta. as únicas informações da rede são: nome e ID de cada amigo, aresta entre cada par de amigos que forem amigos entre si. Nas redes de interação constam arestas dirigidas. As redes eram de pessoas que me mandavam elas de suas contas ou minhas pessoais ou de grupos dos quais participava.
	\item Participa.br: redes de amizade e de interação, texto de postagens, comentários, etc.
	\item Twitter: milhões de tweets permitiram observação contínua de redes de interação (retweet), relacionamentos por vocabulário e hashtags, e padrões do vocabulário em si.
	\item Materiais coletados com entrevistas e oficinas com especialistas.
	\item Estruturas semânticas e dados etiquetados.
\end{itemize}
\end{frame}



\section{Métodos}
\begin{frame}
\frametitle{Métodos}
\begin{itemize}
	\item Medidas circulares
	\item (obtenção das redes de interação)
	\item Setorialização de Erdös
	\item PCA de medidas topológicas
	\item Testes de Kolmogorov-Smirnoff dos textos
	\item Web semântica
	\item Audiovisualização de dados
	\item Considerações tipológicas e humanísticas
\end{itemize}
\end{frame}

\begin{frame}
\frametitle{Medidas circulares}
\end{frame}
\begin{frame}
\frametitle{Redes de interação}
\end{frame}
\begin{frame}
\frametitle{Setorialização de Erdös}
\end{frame}
\begin{frame}
\frametitle{PCA de medidas topológicas}
\end{frame}
\begin{frame}
\frametitle{Kolmogorov-Smirnoff de textos}
\end{frame}
\begin{frame}
\frametitle{Web semântica}
[!Procurar uma notação mais poderosa para os dados ligados.]
\begin{itemize}
	\item Formalização de conceitualizações.
	\item Redes estáveis em nosso tecido social em certas escalas temporais e de pessoas.
		Passível de transições de fase, modificações abruptas em outras.
	\item Permite: análise conjunta de dados de diferentes fontes; inferência por máquina; desenvolvimento conceitual compartilhado.
	\item Recomendação da W3C; padrão acadêmico para dados semânticos etiquetados; melhor formato para entregar os dados para a sociedade como um legado para análise e experimentos.
	\item Pesado e um pouco complicado. Uso de ferramentas como Fuseki/Jena para facilitar os usos.
\end{itemize}

\end{frame}
\begin{frame}
\frametitle{Audiovisualização de dados}
\begin{itemize}
	\item Versinus, imagens, animação abstrata com música, sonificações.
	\item Roteiros automatizados de realização de arte social.
	\item Arte governamental.
\end{itemize}
\end{frame}

\begin{frame}
\frametitle{Considerações tipológicas e humanísticas}
\begin{itemize}
	\item Redes de seres humanos.
	\item Consideração do fator estigmatizante.
	\item Apreciação do meio em que a rede é observada.
	\item Experimentos percolatórios.
	\item Física antropológica.
\end{itemize}
\end{frame}

\section{Resultados}
\begin{frame}
\frametitle{RESULTADOS}
\begin{itemize}
	\item Estabilidade temporal
	\item Diferenciação textual
	\item Iniciação da nuvem brasileira de dados ligados participativos
	\item Aparato em software
	\item Beneficiamento
	\item Ideias ideais (teoria física das ideias)
\end{itemize}
\end{frame}

\begin{frame}
\frametitle{Estabilidade temporal e topológica}
\begin{itemize}
	\item Medidas circulares praticamente iguais para todas as listas e em todas as escalas de segundos a semestres.
	\item Constância dos tamanhos dos setores de Erdös, compatível com as expectativas da literatura. Ainda não achei formalização para esta expectativa e talvez esta seja a primeira.
	\item Estabilidade das componentes principais. Prevalência da centralidade, seguida da simetria e então clusterização dos participantes.
	\item Tipologia não estigmatizante de participante. Tipologia de rede.
\end{itemize}
\end{frame}

\begin{frame}
\frametitle{Diferenciação textual}
\begin{itemize}
	\item O texto produzido por cada setor de Erdös é extremamente diferente um do outro, maiores do que texto produzido por redes diferentes ou mesmo por setores iguais de redes diferentes.
	\item Hubs produzem mais adjetivos. Periféricos mais substantivos, etc.
	\item Formalização de dados ligados a partir de dados relacionais participativos.
	\item Correlações não triviais.
	\item Combinação moderada de medidas topológicas e textuais; prevalência (não extrema) de componentes de texto ou topologia.
	\item Constância da existência - incidência nos textos observados.
\end{itemize}
\end{frame}

\begin{frame}
\frametitle{Dados ligados na web semântica brasileira}
\begin{itemize}
	\item Síntese de ontologias (OWL) e vocabulários (SKOS) de estruturas sociais. OPS, OPa, OPP, Ontologiaa, OCD, OBS, VBS.
	\item Formalização de dados ligados a partir de dados relacionais participativos.
	\item Método de construção de ontologias orientado aos dados.
\end{itemize}
\end{frame}

\begin{frame}
\frametitle{Peças artísticas e mapeamentos sensoriais}
\begin{itemize}
	\item Four hubs dance. Prelúdio social
	\item Versinus.
	\item Outros casos: app online (PHP+python) para imagens do GMANE, sonificações.
	\item Apresentações artísticas: Crânio de Rilke, Freakcoding.
\end{itemize}
\end{frame}

\begin{frame}
\frametitle{Software}
Pacote oficial da linguagem Python (PyPI) para:
	\begin{itemize}
		\item observação das estabilidades topológicas e diferenciações textuais. (Gmane)
		\item Acesso à nuvem de dados participativos brasileiros, jutno aos scritps de triplificação. para análise . (Participation)
		\item Anotação automatizada e semântica de seus próprios dados virtuais provenientes do Facebook, Twitter, Diáspora, IRC, etc. (Social)
		\item Mapeamentos precisos de estruturas sociais em sonoras através da melhor qualidade de síntese. (MASS)
		\item Integração destes dados todos para análise conjunta e navegação semântica enriquecida. Realização de músicas e animações abstratas explorando estas propriedades-chave. (Percolation)
	\end{itemize}
\end{frame}

\begin{frame}
\frametitle{Beneficiamento}
\begin{itemize}
	\item Recomendação de recursos para enriquecimento da navegação semântica.
	\item Experimentos percolatórios.
	\item Fundamentação da origem da lei de potência nestes contextos.
	\item Enriquecimento das tipologias humanas e sociais através do cânone acadêmico para o assunto, que é humanístico.
\end{itemize}
\end{frame}

\begin{frame}
\frametitle{Empréstimos antropológicos}
\begin{itemize}
\item Histórico do termo física antropológica, de Boaz a este trabalho.
\item Aspectos reflexivos, biográficos. Estudo e exposição de si. Diário. Leitura da curtíssima autobiografia.
\item Comparação entre física e antropologia. Considerações sobre uma ciência sólida.
\item \emph{Social physics} do Pentland (co-fundador e diretor do MIT Media Lab).
\end{itemize}
\end{frame}

\begin{frame}
\frametitle{Ideias ideais}
%\includegraphics[scale=.23]{FlowChart.jpg}\\
\begin{itemize}
\item O que é.
\item Como é formada a rede de uma ideia.
\item Livre de escala na forma principal, possui diversos harmônicos.
\item Pelo mesmo raciocínio da origem da propriedade livre de escala, podem ser consideros sensores abstratos.
\item Ideias ideais como unificação das redes sociais formadas por indivíduos e por conceitos.
\item Exemplos de 2002.
\end{itemize}
\end{frame}

\section{Conclusões}
\begin{frame}{Conclusões}
\begin{itemize}
	\item Gradus unificando:
		\begin{itemize}
			\item  apresentação breve e instrumental da área para o indivíduo. Conceitos fundamentais. Paradigmas de redes. Sinonímias, e ambiquidades. Caracterização de redes humanas e beneficiamento para o indivíduo através de experimentos antropológicos, análise e navegação.
				Apêndice com listagens úteis, como medidas, trabalhos de referência, etc.
			\item Questões de física antropológica (ou como esta conceituação se resolver).
			\item Aparato em software, protocolos e dados.
		\end{itemize}
	\item Tipologias com o maior desenvolvimento das análises de estabilidade temporal e diferenciação textual
	\item Legado em software, ontologias e dados
	\item Estou aquém de equipe e qualidade de vida do que acredito adequado. 
	\item Ignorância. Froteira da ilha sempre se ampliando.
\end{itemize}
\end{frame}

\begin{frame}{Próximos passos}
Meu norte seria já poder estudar diretamente a cadeia de sistemas dinâmicos, do social ao universo. Com sua inércia, deve dar para tirar umas boas propriedades relacionando as frequências de sistemas complexos em escalas diferentes. As referências de escala individual ao universo observável devem implicar em propriedades.

Ou seja:
	\begin{itemize}
		\item muita coisa boa para fazer, seria bom delimitar o que vai para pós-doc ou próximo vínculo.
		\item Isso parece depender de alguma previsão de vínculo apropriado e de onde.
		\item Prioridade é permanecer em S. Carlos, de preferência relacionado ao IFSC. Possibilidades na UFSCar, IPRJ, UnB, UFABC.
		\item Trabalhos avulso e relacionamentos com centros culturais como simbióse.
	\end{itemize}
\end{frame}



\begin{frame}{References}
  \begin{thebibliography}{99}
  \bibitem{one}
Anis Das Sharma, Alpa Jain, Kong Yu, " Dynamic Relationship and Event Discvery".
\bibitem{two}
Nguyen Bach and Sameer Badaskar, Presentation on "Survey on Relation Extraction".
\end{thebibliography}
\end{frame}

\begin{frame}
\Large
\begin{center}
 \sc {Obrigado \ldots} 
\end{center}
\end{frame}
%---------------------------------------------------------------------------%
\end{document}

